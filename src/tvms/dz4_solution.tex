
\documentclass[12pt]{article}

\usepackage{iftex}
\ifPDFTeX
  \usepackage[T2A]{fontenc}
  \usepackage[utf8]{inputenc}
  \usepackage[russian]{babel}
\else
  \usepackage{fontspec}
  \setmainfont{DejaVu Serif}
  \usepackage[russian]{babel}
\fi
\usepackage{amsmath, amssymb, amsthm}
\usepackage{geometry}
\geometry{margin=2cm}

\setlength{\parindent}{0pt}
\setlength{\parskip}{6pt}

\begin{document}

\section*{Задача 1}

Исследуем  \emph{СКО} (RMSE):
\[
\mathrm{RMSE}(\hat\theta)=\sqrt{\mathbb{E}_\theta\bigl[(\hat\theta-\theta)^2\bigr]}
=\sqrt{\mathrm{Var}_\theta(\hat\theta)+\bigl(\mathbb{E}_\theta\hat\theta-\theta\bigr)^2}.
\]

\subsection*{План моделирования}
Фиксируем объём выборки $n$ и параметр $\theta$. Для каждой точки сетки по параметру $k$:
\begin{enumerate}
  \item многократно ($B$ раз) генерируем выборку нужного распределения;
  \item считаем оценку $\hat\theta$;
  \item считаем квадратичную ошибку $(\hat\theta-\theta)^2$;
  \item усредняем по $B$ повторениям и берём корень: 
  $\widehat{\mathrm{RMSE}}=\sqrt{\frac1B\sum_{b=1}^B(\hat\theta^{(b)}-\theta)^2}$.
\end{enumerate}
Далее строим график $\widehat{\mathrm{RMSE}}$ по $k$

Для каждого значения $k$ проводим $B$ независимых «экспериментов»:
\begin{enumerate}
  \item генерируем выборку $X_1^{(b)},\dots,X_n^{(b)}$ из заданного распределения при фиксированном $\theta$;
  \item по формуле из пункта задания вычисляем оценку $\hat\theta_k^{(b)}$;
  \item считаем ошибку $e_b=\hat\theta_k^{(b)}-\theta$.
\end{enumerate}

Далее оцениваем СКО (RMSE) по $B$ повторам:
\[
\widehat{\mathrm{RMSE}}(k)
=\sqrt{\frac{1}{B}\sum_{b=1}^B\bigl(\hat\theta_k^{(b)}-\theta\bigr)^2}
=\sqrt{\frac{1}{B}\sum_{b=1}^B e_b^2 }.
\]
Затем строим график зависимости $k \mapsto \widehat{\mathrm{RMSE}}(k)$.

\medskip
\noindent\textbf{Выбор диапазонов $k$ в работе:}
\[
\text{(a)}\ k=2,3,\dots,50;\qquad
\text{(b)}\ k=2,3,\dots,50;\qquad
\text{(c)}\ k=2,3,\dots,20.
\]
(В (c) большие $k$ дают сильно нестабильные оценки, поэтому ограничиваем диапазон.)

\subsection*{(a) $\mathrm{Unif}[0,\theta]$, оценка через квантиль $p=1/k$}

Пусть $X_1,\dots,X_n\sim \mathrm{Unif}[0,\theta]$.
В пункте (a) используется квантиль уровня $p=1/k$.
В каждом повторе сортируем выборку и берём порядковую статистику
\[
\widehat q_{1/k} = X_{(m)}, \qquad m=\left\lceil \frac{n}{k}\right\rceil,
\]
после чего вычисляем оценку
\[
\hat\theta_k = k\,\widehat q_{1/k} = k\,X_{(m)}.
\]
Далее по формуле выше считаем $\widehat{\mathrm{RMSE}}(k)$ и строим график.

\subsection*{(b) $\mathrm{Unif}[0,\theta]$, моментная оценка}

Пусть $X_1,\dots,X_n\sim \mathrm{Unif}[0,\theta]$.
Для каждого $k$ в каждом повторе считаем
\[
\overline{X^k}=\frac{1}{n}\sum_{i=1}^n (X_i)^k,
\qquad
\hat\theta_k=\sqrt[k]{(k+1)\,\overline{X^k}}.
\]
Затем вычисляем $\widehat{\mathrm{RMSE}}(k)$ и строим график.

\subsection*{(c) $\mathrm{Exp}(\theta)$, моментная оценка}

Пусть $X_1,\dots,X_n\sim \mathrm{Exp}(\theta)$, где $\theta$ --- параметр интенсивности
в плотности $f(x)=\theta e^{-\theta x}$, $x\ge 0$.
Для каждого $k$ в каждом повторе считаем
\[
\overline{X^k}=\frac{1}{n}\sum_{i=1}^n (X_i)^k,
\qquad
\hat\theta_k=\sqrt[k]{\frac{k!}{\overline{X^k}}}.
\]
После этого по $B$ повторам получаем $\widehat{\mathrm{RMSE}}(k)$ и строим график.

\subsection*{Качественная интерпретация формы графиков}

\begin{itemize}
  \item \textbf{(a)} При росте $k$ берётся всё меньшая квантиль ($p=1/k$) и затем умножается на $k$,
        поэтому шум усиливается, и $\widehat{\mathrm{RMSE}}(k)$ обычно растёт.
  \item \textbf{(b)} Для равномерного распределения (ограниченного сверху $\theta$) высокие степени не приводят
        к «взрыву» редких больших наблюдений, поэтому $\widehat{\mathrm{RMSE}}(k)$ часто убывает и выходит на плато.
  \item \textbf{(c)} Для экспоненциального распределения из-за неограниченного хвоста величины $X^k$ при больших $k$
        становятся крайне нестабильными, поэтому $\widehat{\mathrm{RMSE}}(k)$ быстро растёт с $k$.
\end{itemize}

\section*{Задача 2}

\subsection*{(a) $X_{(1)}$ или $X_{(n)}$ для $\mathrm{Unif}[-\theta,\theta]$}
Распределение симметрично относительно нуля. Для оценки $\theta$ естественно использовать либо
$\hat\theta_1=-X_{(1)}$, либо $\hat\theta_2=X_{(n)}$. По симметрии
\[
-X_{(1)}\stackrel{d}{=}X_{(n)}\quad\Longrightarrow\quad
\mathrm{MSE}(\hat\theta_1)=\mathrm{MSE}(\hat\theta_2).
\]
Значит, \emph{оценки одинаково эффективны} (имеют одинаковое СКО).

\subsection*{(b) $\overline{X}$ для $\mathrm{Norm}(\theta,1)$ или $\overline{X^2}$ для $\mathrm{Norm}(0,\theta)$}
\begin{itemize}
\item Если $X_i\sim N(\theta,1)$, то $\overline{X}$ несмещённа и
$\mathrm{Var}(\overline{X})=\frac{1}{n}$. Информация Фишера по $\theta$ на одно наблюдение равна 1,
поэтому граница Рао--Крамера равна $1/n$ и $\overline{X}$ \emph{эффективна}.
\item Если $X_i\sim N(0,\theta)$, то $\overline{X^2}$ несмещённа и
$\mathrm{Var}(X^2)=2\theta^2$, значит $\mathrm{Var}(\overline{X^2})=\frac{2\theta^2}{n}$.
Информация Фишера по параметру дисперсии $\theta$ равна $I_n(\theta)=\frac{n}{2\theta^2}$,
граница Рао--Крамера: $1/I_n(\theta)=\frac{2\theta^2}{n}$, то есть $\overline{X^2}$ также \emph{эффективна}.
\end{itemize}
Итого: \emph{в каждой своей модели обе указанные оценки являются эффективными}.

\section*{Задача 3}

Дана плотность
\[
f(x,a)=
\begin{cases}
a e^{-a(x-3)}, & x\ge 3,\\
0, & x<3,
\end{cases}
\qquad a>0.
\]
Положим $Y=X-3\sim\mathrm{Exp}(a)$.

\subsection*{Информация Фишера}
Лог-правдоподобие для одного наблюдения ($x\ge3$):
\[
\ell(a;x)=\ln a-a(x-3).
\]
Тогда
\[
\frac{\partial \ell}{\partial a}=\frac1a-(x-3),\qquad
\frac{\partial^2 \ell}{\partial a^2}=-\frac{1}{a^2}.
\]
Следовательно,
\[
I_1(a)=-\mathbb{E}\frac{\partial^2 \ell}{\partial a^2}=\frac{1}{a^2},\qquad
I_n(a)=\frac{n}{a^2}.
\]
Граница Рао--Крамера (для несмещённых): $\mathrm{Var}(\hat a)\ge \frac{1}{I_n(a)}=\frac{a^2}{n}$.

\subsection*{(a) ММП}
Лог-правдоподобие выборки:
\[
\ell_n(a)=n\ln a-a\sum_{i=1}^n (x_i-3).
\]
Условие максимума $\ell_n'(a)=0$ даёт
\[
\hat a_{\mathrm{MLE}}=\frac{n}{\sum_{i=1}^n (x_i-3)}=\frac{1}{\overline{X}-3}.
\]
Из общих свойств ММП для регулярных моделей:
\[
\sqrt{n}\,(\hat a_{\mathrm{MLE}}-a)\ \xrightarrow{d}\ N\Bigl(0,\ \frac{1}{I_1(a)}\Bigr)=N(0,a^2),
\]
то есть $\mathrm{AsyVar}(\hat a_{\mathrm{MLE}})=\frac{a^2}{n}=1/I_n(a)$.
Следовательно, \emph{$\hat a_{\mathrm{MLE}}$ является $R$-эффективной}.

\subsection*{(b) Метод моментов}

\paragraph{1) Для $g(x)=e^{-x}$.}
\[
\mathbb{E}[e^{-X}]=e^{-3}\,\mathbb{E}[e^{-Y}]=e^{-3}\frac{a}{a+1}.
\]
Пусть $\overline{g}=\frac1n\sum_{i=1}^n e^{-X_i}$. Уравнение моментов:
\[
e^{-3}\frac{a}{a+1}=\overline{g}\quad\Longleftrightarrow\quad
\frac{a}{a+1}=e^3\overline{g}=:T.
\]
Отсюда моментная оценка:
\[
\hat a_{1}=\frac{T}{1-T}=\frac{e^3\overline{g}}{1-e^3\overline{g}}.
\]
Найдём асимптотическую дисперсию.
Имеем $U=e^{-X}=e^{-3}e^{-Y}$, где $Y\sim\mathrm{Exp}(a)$, поэтому
\[
\mathbb{E}U=e^{-3}\frac{a}{a+1},\qquad
\mathbb{E}U^2=e^{-6}\frac{a}{a+2},
\]
\[
\mathrm{Var}(U)=e^{-6}\Bigl(\frac{a}{a+2}-\frac{a^2}{(a+1)^2}\Bigr).
\]
Так как $T=e^3\overline{U}$, то
\[
\mathrm{Var}(T)=e^6\mathrm{Var}(\overline{U})
=\frac{1}{n}\Bigl(\frac{a}{a+2}-\frac{a^2}{(a+1)^2}\Bigr).
\]
Функция $h(t)=\frac{t}{1-t}$ имеет производную $h'(t)=\frac{1}{(1-t)^2}$,
а при $t_0=\frac{a}{a+1}$ получаем $h'(t_0)=(a+1)^2$.
По дельта-методу:
\[
\mathrm{AsyVar}(\hat a_1)= (h'(t_0))^2\,\mathrm{Var}(T)
=\frac{(a+1)^4}{n}\Bigl(\frac{a}{a+2}-\frac{a^2}{(a+1)^2}\Bigr)
=\frac{a(a+1)^2}{n(a+2)}.
\]
Сравним с $1/I_n(a)=a^2/n$:
\[
\frac{\mathrm{AsyVar}(\hat a_1)}{a^2/n}
=\frac{(a+1)^2}{a(a+2)}>1.
\]
Значит, \emph{$\hat a_1$ не является $R$-эффективной}.

\paragraph{2) Для $g(x)=x^2$.}
\[
\mathbb{E}[X^2]=\mathbb{E}[(3+Y)^2]=9+6\mathbb{E}Y+\mathbb{E}Y^2
=9+\frac{6}{a}+\frac{2}{a^2}.
\]
Пусть $S_2=\overline{X^2}=\frac1n\sum X_i^2$. Уравнение моментов:
\[
S_2=9+\frac{6}{a}+\frac{2}{a^2}.
\]
Обозначив $t=\frac1a$, получаем квадратное уравнение
$2t^2+6t+(9-S_2)=0$ и (берём положительный корень) 
\[
\hat a_2=\left(\frac{-6+\sqrt{8S_2-36}}{4}\right)^{-1}.
\]
Для асимптотической дисперсии используем дельта-метод в неявном виде.
Пусть $m(a)=\mathbb{E}[X^2]=9+\frac{6}{a}+\frac{2}{a^2}$, тогда $m' (a)= -\frac{6}{a^2}-\frac{4}{a^3}$.
Кроме того, для $Y\sim\mathrm{Exp}(a)$ имеем $\mathbb{E}Y^r=\frac{r!}{a^r}$, откуда
\[
\mathbb{E}[X^4]=\mathbb{E}[(3+Y)^4]
=81+\frac{108}{a}+\frac{108}{a^2}+\frac{72}{a^3}+\frac{24}{a^4}.
\]
Тогда
\[
\mathrm{Var}(X^2)=\mathbb{E}[X^4]-\bigl(\mathbb{E}[X^2]\bigr)^2
=\frac{36}{a^2}+\frac{48}{a^3}+\frac{20}{a^4}.
\]
По ЦПТ и дельта-методу:
\[
\mathrm{AsyVar}(\hat a_2)=\frac{\mathrm{Var}(X^2)}{n\,(m'(a))^2}
=\frac{a^2}{n}\cdot\frac{9a^2+12a+5}{9a^2+12a+4}.
\]
Так как $\frac{9a^2+12a+5}{9a^2+12a+4}>1$, то \emph{$\hat a_2$ также не $R$-эффективна}.

\section*{Задача 4}

\subsection*{I. $\mathrm{Unif}[0,\theta]$, статистика $X_{(n)}$}

Пусть $X_1,\dots,X_n\sim\mathrm{Unif}[0,\theta]$, $X_{(n)}=\max X_i$.
Тогда для $0\le x\le\theta$:
\[
\mathbb{P}(X_{(n)}\le x)=\left(\frac{x}{\theta}\right)^n.
\]
Эквивалентно, $U=X_{(n)}/\theta$ имеет CDF $F_U(u)=u^n$, $u\in[0,1]$.

\paragraph{1) Односторонний точный ДИ уровня значимости $\alpha$ (доверие $1-\alpha$).}
Ищем $c$ из
\[
\mathbb{P}\{\theta\le cX_{(n)}\}=1-\alpha.
\]
Имеем
\[
\mathbb{P}\{\theta\le cX_{(n)}\}
=\mathbb{P}\left\{X_{(n)}\ge \frac{\theta}{c}\right\}
=1-\left(\frac{1}{c}\right)^n.
\]
Отсюда $c=\alpha^{-1/n}$ и
\[
I_1:\quad \theta\in\bigl[X_{(n)},\ \alpha^{-1/n}X_{(n)}\bigr]\quad\text{с вероятностью }1-\alpha.
\]

\paragraph{2) Двусторонний точный ДИ уровня значимости $\alpha$.}
Пусть
\[
\mathbb{P}\Bigl\{u_1\le \frac{X_{(n)}}{\theta}\le u_2\Bigr\}=1-\alpha,
\qquad 
\mathbb{P}\Bigl\{\frac{X_{(n)}}{\theta}<u_1\Bigr\}=\frac{\alpha}{2},
\quad 
\mathbb{P}\Bigl\{\frac{X_{(n)}}{\theta}>u_2\Bigr\}=\frac{\alpha}{2}.
\]
Так как $F_U(u)=u^n$, получаем
\[
u_1=\left(\frac{\alpha}{2}\right)^{1/n},\qquad
u_2=\left(1-\frac{\alpha}{2}\right)^{1/n}.
\]
Неравенство $u_1\le X_{(n)}/\theta\le u_2$ эквивалентно
\[
\frac{X_{(n)}}{u_2}\le \theta \le \frac{X_{(n)}}{u_1}.
\]
Следовательно,
\[
I_2:\quad 
\theta\in\left[\frac{X_{(n)}}{(1-\alpha/2)^{1/n}},\ \frac{X_{(n)}}{(\alpha/2)^{1/n}}\right]
\quad\text{с вероятностью }1-\alpha.
\]

\paragraph{(*) Длина интервалов и асимптотика.}
Длины:
\[
|I_1|=X_{(n)}\bigl(\alpha^{-1/n}-1\bigr),\qquad
|I_2|=X_{(n)}\left((\alpha/2)^{-1/n}-(1-\alpha/2)^{-1/n}\right).
\]
Так как $\mathbb{E}X_{(n)}=\frac{n}{n+1}\theta$ и при $n\to\infty$ $a^{-1/n}=e^{-\ln(a)/n}=1-\frac{\ln a}{n}+o(1/n)$,
то
\[
\mathbb{E}|I_1|\sim \theta\cdot\frac{-\ln\alpha}{n},\qquad
\mathbb{E}|I_2|\sim \theta\cdot\frac{-\ln(\alpha/2)+\ln(1-\alpha/2)}{n}.
\]
Оба интервала \emph{сужаются как $O(1/n)$}, причём $I_1$ короче (у $I_2$ дополнительно появляется вклад $\ln 2$).

\subsection*{II. Точные ДИ по $X_{(1)}$}

\paragraph{(a) $X_i\sim\mathrm{Unif}[\theta,\theta+1]$.}
Пусть $Y_i=X_i-\theta\sim\mathrm{Unif}[0,1]$. Тогда $Y_{(1)}=\min Y_i$ и
\[
\mathbb{P}(Y_{(1)}>y)=(1-y)^n,\quad y\in[0,1].
\]
Выберем $c$ из $\mathbb{P}(Y_{(1)}\le c)=1-\alpha$:
\[
1-(1-c)^n=1-\alpha \quad\Rightarrow\quad c=1-\alpha^{1/n}.
\]
Так как $Y_{(1)}=X_{(1)}-\theta$, получаем точный интервал:
\[
\theta\in\Bigl[X_{(1)}-(1-\alpha^{1/n}),\ X_{(1)}\Bigr]\quad\text{с вероятностью }1-\alpha.
\]

\paragraph{(b) $X_i\sim\mathrm{Unif}[\theta,2\theta]$.}
Положим $W_i=X_i/\theta\sim\mathrm{Unif}[1,2]$. Тогда $W_{(1)}=X_{(1)}/\theta$ и
\[
\mathbb{P}(W_{(1)}>w)=(2-w)^n,\quad w\in[1,2].
\]
Выбираем $c$ из $\mathbb{P}(W_{(1)}\le c)=1-\alpha$:
\[
1-(2-c)^n=1-\alpha\quad\Rightarrow\quad c=2-\alpha^{1/n}.
\]
Неравенство $W_{(1)}\le c$ эквивалентно $\theta\ge X_{(1)}/c$. Также всегда $\theta\le X_{(1)}$.
Итак,
\[
\theta\in\Bigl[\frac{X_{(1)}}{2-\alpha^{1/n}},\ X_{(1)}\Bigr]\quad\text{с вероятностью }1-\alpha.
\]

\section*{Задача 5}

Пусть $X_1,\dots,X_n\sim N(0,\theta)$, где $\theta>0$ --- дисперсия (мат. ожидание известно и равно 0).
Пусть уровень доверия равен $\gamma\in(0,1)$ (в терминах значимости $\alpha$: $\gamma=1-\alpha$).

\subsection*{(a) Интервал по статистике $\overline{X^2}$ (эквивалентно $\sum X_i^2$)}
Известно, что
\[
\frac{\sum_{k=1}^n X_k^2}{\theta}\sim \chi^2_n.
\]
Пусть $\lambda_p$ --- квантиль порядка $p$ распределения $\chi^2_n$.
Тогда
\[
\mathbb{P}\left\{\lambda_{(1-\gamma)/2}\le \frac{\sum X_k^2}{\theta}\le \lambda_{(1+\gamma)/2}\right\}=\gamma.
\]
Инвертируя неравенство по $\theta$, получаем точный ДИ:
\[
\theta\in\left(\frac{\sum_{k=1}^n X_k^2}{\lambda_{(1+\gamma)/2}},\ 
\frac{\sum_{k=1}^n X_k^2}{\lambda_{(1-\gamma)/2}}\right)
\quad\text{с вероятностью } \gamma,
\]
что совпадает с формулой из условия.

\emph{Длина при росте $n$.}
Длина равна
\[
L_a=\sum X_k^2\left(\frac{1}{\lambda_{(1-\gamma)/2}}-\frac{1}{\lambda_{(1+\gamma)/2}}\right).
\]
При $n\to\infty$ квантили $\chi^2_n$ удовлетворяют
$\lambda_p=n+z_p\sqrt{2n}+o(\sqrt{n})$, поэтому $L_a=O_\mathbb{P}(\theta/\sqrt{n})$,
то есть интервал \emph{сужается как $1/\sqrt{n}$}.

\subsection*{(b) Интервал по статистике $\overline{X}^{\,2}$}
Так как $\overline{X}\sim N\!\left(0,\frac{\theta}{n}\right)$, то
\[
\frac{\sqrt{n}\,\overline{X}}{\sqrt{\theta}}\sim N(0,1)
\quad\Rightarrow\quad
\frac{n\overline{X}^{\,2}}{\theta}\sim \chi^2_1.
\]
Для $\chi^2_1$ квантиль можно выразить через нормальные квантили:
если $Z\sim N(0,1)$, то $Z^2\sim \chi^2_1$ и
\[
\chi^2_{1,p}=z^2_{(p+1)/2}.
\]
Поэтому
\[
\chi^2_{1,(1+\gamma)/2}=z^2_{(3+\gamma)/4},\qquad
\chi^2_{1,(1-\gamma)/2}=z^2_{(3-\gamma)/4}.
\]
Отсюда точный ДИ:
\[
\theta\in\left(\frac{n\overline{X}^{\,2}}{z^2_{(3+\gamma)/4}},\ 
\frac{n\overline{X}^{\,2}}{z^2_{(3-\gamma)/4}}\right)
\quad\text{с вероятностью }\gamma,
\]
что совпадает с формулой из условия.

\emph{Длина при росте $n$.}
Так как $\frac{n\overline{X}^{\,2}}{\theta}\sim\chi^2_1$ и распределение не зависит от $n$,
то $n\overline{X}^{\,2}$ имеет типичный порядок $\theta$ \emph{для любого $n$}.
Следовательно, длина $L_b$ \emph{не стремится к нулю при $n\to\infty$} (остаётся порядка $\theta$):
интервал использует лишь одну степень свободы (среднее), поэтому крайне неэффективен
для оценивания дисперсии по сравнению с пунктом (a).

\section*{Задача 6}

\subsection*{I. $\mathrm{Unif}[0,\theta]$}

\paragraph{(a) Асимптотический ДИ по $X_{(n)}$.}
Известно предельное распределение:
\[
V_n:=\frac{n(\theta-X_{(n)})}{\theta}\ \xrightarrow{d}\ \mathrm{Exp}(1),
\qquad
\mathbb{P}(\mathrm{Exp}(1)\le t)=1-e^{-t}.
\]
Пусть $q_p=-\ln(1-p)$ --- квантиль порядка $p$ распределения $\mathrm{Exp}(1)$.
Тогда для двустороннего уровня $1-\alpha$:
\[
\mathbb{P}\{q_{\alpha/2}\le V_n\le q_{1-\alpha/2}\}\approx 1-\alpha.
\]
Имеем $q_{\alpha/2}=-\ln(1-\alpha/2)$, $q_{1-\alpha/2}=-\ln(\alpha/2)$.
Неравенство $q_1\le V_n\le q_2$ эквивалентно
\[
q_1\le n\Bigl(1-\frac{X_{(n)}}{\theta}\Bigr)\le q_2
\quad\Longleftrightarrow\quad
1-\frac{q_2}{n}\le \frac{X_{(n)}}{\theta}\le 1-\frac{q_1}{n}.
\]
Инвертируя, получаем асимптотический ДИ:
\[
\theta\in\left[\frac{X_{(n)}}{1-q_1/n},\ \frac{X_{(n)}}{1-q_2/n}\right]
\quad\text{(приблизительно с вероятностью }1-\alpha),
\]
где $q_1=-\ln(1-\alpha/2)$, $q_2=-\ln(\alpha/2)$.
(При практическом использовании требуется $n>q_2$.)

\paragraph{(b) Асимптотические нормальные оценки $\hat\theta_1=2\overline{X}$ и $\hat\theta_2=\sqrt{3\,\overline{X^2}}$.}
Для $\mathrm{Unif}[0,\theta]$:
\[
\mathbb{E}X=\frac{\theta}{2},\quad \mathrm{Var}(X)=\frac{\theta^2}{12}
\ \Rightarrow\ 
\mathrm{Var}(\hat\theta_1)=4\frac{\theta^2}{12n}=\frac{\theta^2}{3n}.
\]
Значит,
\[
\hat\theta_1 \approx N\!\left(\theta,\frac{\theta^2}{3n}\right).
\]
А также $\mathbb{E}X^2=\theta^2/3$, $\mathrm{Var}(X^2)=\theta^4\left(\frac15-\frac19\right)=\frac{4\theta^4}{45}$.
По дельта-методу для $g(m)=\sqrt{3m}$ получаем
\[
\mathrm{Var}(\hat\theta_2)\approx \frac{\theta^2}{5n},
\qquad
\hat\theta_2 \approx N\!\left(\theta,\frac{\theta^2}{5n}\right).
\]
Отсюда асимптотические ДИ уровня $1-\alpha$:
\[
\theta\in\left[\hat\theta_1\Bigl(1-\frac{z_{1-\alpha/2}}{\sqrt{3n}}\Bigr),\ 
\hat\theta_1\Bigl(1+\frac{z_{1-\alpha/2}}{\sqrt{3n}}\Bigr)\right],
\]
\[
\theta\in\left[\hat\theta_2\Bigl(1-\frac{z_{1-\alpha/2}}{\sqrt{5n}}\Bigr),\ 
\hat\theta_2\Bigl(1+\frac{z_{1-\alpha/2}}{\sqrt{5n}}\Bigr)\right].
\]
Их асимптотические длины пропорциональны стандартным ошибкам:
\[
|I_1|\sim \frac{2z_{1-\alpha/2}\theta}{\sqrt{3n}},\qquad
|I_2|\sim \frac{2z_{1-\alpha/2}\theta}{\sqrt{5n}}.
\]
Так как $\frac{1}{\sqrt{5}}<\frac{1}{\sqrt{3}}$, то \emph{интервал на основе $\hat\theta_2$ асимптотически короче}
(в $\sqrt{3/5}$ раз).

\subsection*{II. $X_i\sim\mathrm{Exp}(\beta)$, асимптотические ДИ для $\beta$}

Пусть $\beta>0$ --- параметр интенсивности. Даны оценки
\[
\hat\beta_1=\frac{1}{\overline{X}},\qquad
\hat\beta_2=\sqrt{\frac{2}{\overline{X^2}}}.
\]

Так как $\mathbb{E}X=\frac{1}{\beta}$, $\mathrm{Var}(X)=\frac{1}{\beta^2}$, то по дельта-методу
\[
\hat\beta_1 \approx N\!\left(\beta,\frac{\beta^2}{n}\right).
\]
Далее, $\mathbb{E}X^2=\frac{2}{\beta^2}$, $\mathbb{E}X^4=\frac{24}{\beta^4}$, значит
$\mathrm{Var}(X^2)=\frac{20}{\beta^4}$ и по дельта-методу для $h(m)=\sqrt{2/m}$:
\[
\hat\beta_2 \approx N\!\left(\beta,\frac{5\beta^2}{4n}\right).
\]
Отсюда асимптотические ДИ уровня $1-\alpha$:
\[
\beta\in\left[\hat\beta_1\Bigl(1-\frac{z_{1-\alpha/2}}{\sqrt{n}}\Bigr),\ 
\hat\beta_1\Bigl(1+\frac{z_{1-\alpha/2}}{\sqrt{n}}\Bigr)\right],
\]
\[
\beta\in\left[\hat\beta_2\Bigl(1-\frac{z_{1-\alpha/2}}{\sqrt{5n/4}}\Bigr),\ 
\hat\beta_2\Bigl(1+\frac{z_{1-\alpha/2}}{\sqrt{5n/4}}\Bigr)\right].
\]
Сравнение длин:
\[
\frac{|J_2|}{|J_1|}\ \to\ \sqrt{\frac{5}{4}}>1,
\]
то есть \emph{интервал на основе $\hat\beta_1$ асимптотически короче}.

\end{document}

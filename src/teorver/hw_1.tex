
% Preamble
\documentclass[11pt]{article}


\usepackage[T2A]{fontenc}
\usepackage[utf8]{inputenc}
\usepackage[russian]{babel}
\usepackage{amsmath,amssymb,mathtools}
\usepackage{geometry}
\geometry{margin=1in}
\usepackage{hyperref}
\usepackage{amsfonts}
\usepackage{amstext}

\newcommand{\PP}{\mathbb{P}}

% Document
\begin{document}
    \begin{center}
    {\LARGE Домашнее задание №1}
        \\[4pt]
        {\small (Арапов А.А. ВБИБ 2025)}
    \end{center}


    \section*{Задача 1}\label{sec:-1}

    \textbf{Шаг 1. Найдем сколько 5ок с любым количеством тузов}

    Ровно 3 красные и ровно 1 черва, т.е. 1 черва, 2 бубны, 2 чёрные.
    Таких комбинаций:
    \[N_0=\binom{13}{1}\binom{13}{2}\binom{26}{2}=329\,550.\]

    \textbf{Шаг 2. Вычтем случаи с $\ge 3$ тузами.}

    \emph{(1) Ровно 4 туза.} Тогда пятая карта должна быть бубновой, не тузом: $12$ вариантов.
    \[
        N_1 = \binom{12}{2}
    \]

    \emph{(2) Ровно 3 туза.}
    \begin{itemize}
        \item Нет $A\diamondsuit$: есть $A\heartsuit,A\clubsuit,A\spadesuit$; нужно добавить $2$ бубны без тузов: $\binom{12}{2}=66$.
        \item Нет $A\heartsuit$: есть $A\diamondsuit,A\clubsuit,A\spadesuit$; т.е. бубна без туза и черви без туза: $\binom{12}{1}\cdot\binom{12}{1}=144$.
        \item Нет $A\clubsuit$: есть $A\heartsuit,A\diamondsuit,A\spadesuit$; бубна без туза и черная без туза: $\binom{12}{1}\cdot\binom{24}{1}=288$.
        \item Нет $A\spadesuit$: симметрично предыдущему: $\binom{12}{1}\cdot\binom{24}{1}=288$.
    \end{itemize}
    Итого для 3 тузов: $ N_2 = 66+144+288+288=786$.

    \smallskip
    Суммарно с 3 или 4 тузами:
    \[
        N_3=786+12=798.
    \]

    \textbf{Ответ.} Благоприятных пятёрок
    \[
        N = N_0-N_3=329\,550-798=328\,752.
    \]
    Вероятность
    \[
        \PP=\frac{N}{\binom{52}{5}}=\frac{328\,752}{2\,598\,960}
        =\frac{6\,849}{54\,145}\approx 0.12649\;\;(\approx 12.649\%).
    \]


    \section*{Задача 2}\label{sec:0}
\end{document}
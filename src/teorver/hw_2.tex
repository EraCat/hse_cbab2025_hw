%! Author = alexx
%! Date = 24.10.2025

\documentclass[12pt]{article}

\usepackage[utf8]{inputenc}
\usepackage[russian]{babel}
\usepackage{amsmath,amssymb,amsthm, mathtools}
\usepackage{geometry}
\geometry{margin=2cm}

\DeclareMathOperator{\Var}{\mathbb{D}}
\DeclareMathOperator{\Cov}{Cov}
\DeclareMathOperator{\Corr}{Corr}
\newcommand{\E}{\mathbb{E}}
\newcommand{\Pbb}{\mathbb{P}}

\begin{document}

    \section*{Задача 0}

    \subsection*{(a)}
    Пусть случайная величина $X$ имеет моменты
    \[
        \E X = 1,\quad
        \E X^2 = 2,\quad
        \E X^3 = 3,\quad
        \E X^4 = 4.
    \]

    Проверим совместимость этих условий.

    Рассмотрим случайную величину $Y = X^2$. Из неравенства Коши--Буняковского (а эквивалентно, из того, что $\Var(Y)\ge 0$) следует
    \[
        \E Y^2 \ge (\E Y)^2.
    \]
    Здесь
    \[
        \E Y = \E X^2 = 2,\qquad
        \E Y^2 = \E X^4 = 4,
    \]
    поэтому
    \[
        4 = \E X^4 = \E Y^2 \ge (\E Y)^2 = (\E X^2)^2 = 4.
    \]
    То есть равенство достигается. Равенство в Коши--Буняковском достигается тогда и только тогда, когда $Y$ п.н. пропорциональна константе, то есть $Y$ (а значит $X^2$) является константой почти наверное. Следовательно,
    \[
        X^2 \equiv 2 \quad \text{п.н.}
    \]
    Отсюда $X^3 = X\cdot X^2 = 2X$, и
    \[
        \E X^3 = \E(2X) = 2\,\E X = 2\cdot 1 = 2.
    \]
    Но по условию требуется $\E X^3 = 3$. Противоречие.

    \medskip
    \noindent
    \textbf{Вывод:} случайной величины $X$ с заданными моментами не существует.

    \subsection*{(b)}
    Пусть $X,Y,Z$ --- независимые случайные величины. Рассмотрим $X+Z$ и $Y+Z$.

    Проверим, независимы ли они. Для этого найдём ковариацию:
    \[
        \Cov(X+Z,\; Y+Z)
        = \Cov(X,Y) + \Cov(X,Z) + \Cov(Z,Y) + \Var(Z).
    \]
    Так как $X,Y,Z$ попарно независимы, $\Cov(X,Y)=\Cov(X,Z)=\Cov(Y,Z)=0$. Тогда
    \[
        \Cov(X+Z,\; Y+Z) = \Var(Z).
    \]
    Если $\Var(Z)>0$, то ковариация не равна нулю, значит $X+Z$ и $Y+Z$ не могут быть независимы (независимые случайные величины имеют нулевую ковариацию).

    Единственный случай, когда $\Var(Z)=0$, это вырожденно-константный $Z\equiv c$. Тогда $X+Z = X+c$ и $Y+Z = Y+c$, и они независимы как сдвиги независимых $X$ и $Y$.

    \medskip
    \noindent
    \textbf{Вывод:} вообще говоря, $X+Z$ и $Y+Z$ не являются независимыми. Они независимы только в вырожденном случае, когда $Z$ почти наверное константа.


    \section*{Задача 1}

    Пусть $X_1,\dots,X_n$ --- независимые одинаково распределённые случайные величины (н.о.р.с.в.). Введём обозначение:
    \[
        S_{a,m,k} = \sum_{i=m}^{k} X_i^{\,a}.
    \]

    Будем использовать следующие моменты одной типичной величины $X$:
    \begin{gather*}
        \mu = \E X,\qquad
        \mu_2 = \E X^2,\qquad
        \sigma^2 = \Var(X) = \mu_2 - \mu^2,\\
        \mu_3 = \E X^3,\qquad
        \mu_4 = \E X^4.\\
    \end{gather*}

    \subsection*{(a) Коэффициент корреляции между $S_{1,1,k}$ и $S_{1,2,k}$}

    Здесь
    \[
        S_{1,1,k} = \sum_{i=1}^k X_i,\qquad
        S_{1,2,k} = \sum_{i=2}^k X_i.
    \]
    Общие слагаемые у них: индексы $i=2,\dots,k$, то есть $(k-1)$ штук.

    Так как все $X_i$ независимы и одинаково распределены,
    \[
        \Var(S_{1,1,k}) = k\,\sigma^2,\qquad
        \Var(S_{1,2,k}) = (k-1)\,\sigma^2.
    \]
    Ковариация двух сумм равна сумме ковариаций попарно совпадающих индексов:
    \[
        \Cov(S_{1,1,k}, S_{1,2,k})
        = \sum_{i=2}^{k} \Var(X_i)
        = (k-1)\,\sigma^2.
    \]
    Тогда коэффициент корреляции
    \[
        \rho
        = \frac{\Cov(S_{1,1,k}, S_{1,2,k})}{\sqrt{\Var(S_{1,1,k})\,\Var(S_{1,2,k})}}
        = \frac{(k-1)\sigma^2}{\sqrt{k\sigma^2 \cdot (k-1)\sigma^2}}
        = \sqrt{\frac{k-1}{k}}.
    \]

    \subsection*{(b) Коэффициент корреляции между $S_{1,1,k}$ и $S_{1,j,n}$}

    Пусть $n>k>j>1$.
    \[
        S_{1,1,k} = \sum_{i=1}^k X_i,\qquad
        S_{1,j,n} = \sum_{i=j}^n X_i.
    \]
    Общие индексы: $i=j,j+1,\dots,k$, то есть $(k-j+1)$ штук.

    Тогда
    \[
        \Cov(S_{1,1,k}, S_{1,j,n})
        = \sum_{i=j}^{k} \Var(X_i)
        = (k-j+1)\,\sigma^2,
    \]
    \[
        \Var(S_{1,1,k}) = k\,\sigma^2,\qquad
        \Var(S_{1,j,n}) = (n-j+1)\,\sigma^2.
    \]
    Следовательно,
    \[
        \rho
        = \frac{(k-j+1)\sigma^2}{\sqrt{k\sigma^2 \cdot (n-j+1)\sigma^2}}
        = \frac{k-j+1}{\sqrt{k\,(n-j+1)}}.
    \]

    \subsection*{(c)$^\ast$ Коэффициент корреляции между $S_{1,1,k}$ и $S_{2,1,j}$}

    Теперь
    \[
        S_{1,1,k} = \sum_{i=1}^k X_i,\qquad
        S_{2,1,j} = \sum_{i=1}^j X_i^2.
    \]

    Сначала найдём ковариацию $\Cov(X_i, X_i^2)$:
    \[
        \Cov(X_i, X_i^2)
        = \E[X_i^3] - \E[X_i]\E[X_i^2]
        = \mu_3 - \mu\,\mu_2.
    \]
    Если $i\neq r$, то $X_i$ и $X_r^2$ независимы, значит ковариация равна нулю.

    Пересекающиеся индексы у сумм $S_{1,1,k}$ и $S_{2,1,j}$: $i=1,\dots,\min(k,j)$. Значит
    \[
        \Cov(S_{1,1,k}, S_{2,1,j})
        = \sum_{i=1}^{\min(k,j)} \Cov(X_i, X_i^2)
        = \min(k,j)\,(\mu_3 - \mu\,\mu_2).
    \]

    Дисперсии:
    \[
        \Var(S_{1,1,k}) = k\,\sigma^2.
    \]
    Для $S_{2,1,j}$:
    \[
        \Var(S_{2,1,j})
        = \sum_{i=1}^j \Var(X_i^2)
        = j \,\Var(X^2)
        = j\left( \E X^4 - (\E X^2)^2 \right)
        = j(\mu_4 - \mu_2^2).
    \]

    Отсюда коэффициент корреляции
    \[
        \rho
        = \frac{\Cov(S_{1,1,k}, S_{2,1,j})}{\sqrt{\Var(S_{1,1,k})\Var(S_{2,1,j})}}
        = \frac{ \min(k,j)\,(\mu_3 - \mu\,\mu_2)}
        {\sqrt{ k\sigma^2 \cdot j(\mu_4 - \mu_2^2) }}.
    \]


    \section*{Задача 2}

    Задана совместная плотность $(X,Y)$:
    \[
        f_{X,Y}(x,y)=
        \begin{cases}
            c\,(x+y), & 0 \le y < x \le 1,\\
            0, & \text{иначе.}
        \end{cases}
    \]

    \subsection*{(a) Найти константу $c$ и функцию распределения $F_{X,Y}(x,y)$}

    \textbf{Шаг 1. Нормировка.}
    Требуем $\iint f_{X,Y}(x,y)\,dy\,dx = 1$. Интегрируем по области $0\le y < x \le 1$:
    \[
        1 = \int_0^1 \int_0^x c(x+y)\,dy\,dx.
    \]
    Внутренний интеграл:
    \[
        \int_0^x (x+y)\,dy = x^2 + \frac{x^2}{2} = \frac{3}{2}x^2.
    \]
    Тогда
    \[
        1 = \int_0^1 c \cdot \frac{3}{2}x^2\,dx
        = c\cdot \frac{3}{2}\cdot \frac{1}{3}
        = \frac{c}{2}.
    \]
    Отсюда $c=2$. Значит
    \[
        f_{X,Y}(x,y) =
        \begin{cases}
            2(x+y), & 0 \le y < x \le 1,\\
            0, & \text{иначе.}
        \end{cases}
    \]

    \medskip

    \textbf{Шаг 2. Функция распределения.}
    Определим
    \[
        F_{X,Y}(x,y) = \Pbb(X \le x,\; Y \le y).
    \]

    Общая область ненулевой плотности: $0 \le y' < x' \le 1$.

    Пусть сначала $0 \le x \le 1$, $0 \le y \le 1$.

    \textbf{Случай 1: $0 \le x \le y \le 1$.}
    Тогда условие $Y \le y$ не является ограничением, так как для всех $x' \le x$ и $y' < x'$ мы имеем $y' \le x' \le x \le y$. Значит
    \[
        F(x,y)
        = \int_{0}^{x} \int_{0}^{x'} 2(x'+y')\,dy'\,dx'.
    \]
    Внутренний интеграл:
    \[
        \int_0^{x'} 2(x'+y')\,dy' = 2\left[x'y' + \frac{y'^2}{2}\right]_{0}^{x'}
        = 2\left[x'^2 + \frac{x'^2}{2}\right]
        = 3x'^2.
    \]
    Тогда
    \[
        F(x,y)
        = \int_0^{x} 3x'^2\,dx'
        = x^3.
    \]

    \textbf{Случай 2: $0 \le y \le x \le 1$.}
    Теперь условие $Y \le y$ ``отрезает'' область по горизонтали. Тогда
    \[
        F(x,y)
        = \int_{0}^{y} \int_{0}^{x'} 2(x'+y')\,dy'\,dx'
        + \int_{y}^{x} \int_{0}^{y} 2(x'+y')\,dy'\,dx'.
    \]
    Первый интеграл:
    \[
        \int_0^{y} \left[ \int_0^{x'} 2(x'+y')\,dy' \right] dx'
        = \int_0^{y} 3x'^2\,dx'
        = y^3.
    \]
    Второй интеграл: при фиксированном $x'$ от $y'=0$ до $y$,
    \[
        \int_0^{y} 2(x'+y')\,dy'
        = 2\left[x'y + \frac{y^2}{2}\right]
        = 2x'y + y^2.
    \]
    Тогда
    \[
        \int_y^{x} (2x'y + y^2)\,dx'
        = y \int_y^x 2x'\,dx' + y^2(x-y)
        = y(x^2 - y^2) + y^2(x-y)
        = yx^2 + y^2 x - 2y^3.
    \]
    Складывая с $y^3$:
    \[
        F(x,y)
        = y^3 + \bigl(yx^2 + y^2 x - 2y^3\bigr)
        = yx^2 + y^2 x - y^3.
    \]

    Теперь опишем $F(x,y)$ на всей плоскости.

    \[
        F_{X,Y}(x,y)
        =
        \begin{cases}
            0, & x \le 0 \ \text{или}\ y \le 0,\\[6pt]
            x^3, & 0 \le x \le 1,\ x \le y,\\[6pt]
            yx^2 + y^2 x - y^3,
            & 0 \le y \le x \le 1,\\[6pt]
            y + y^2 - y^3, & x \ge 1,\ 0 \le y \le 1,\\[6pt]
            x^3, & 0 \le x \le 1,\ y \ge 1,\\[6pt]
            1, & x \ge 1,\ y \ge 1~.
        \end{cases}
    \]

    Здесь формулы для случаев $x \ge 1$ или $y \ge 1$
    получены из монотонности $F$ и того, что область плотности ограничена $x \le 1$, $y \le 1$.

    \subsection*{(b) Коэффициент корреляции $\Corr(X,Y)$}

    Сначала найдём все необходимые моменты.

    \textbf{Матожидания.}
    \[
        \E X
        = \int_0^1 \int_0^x x \cdot 2(x+y)\,dy\,dx
        = 2 \int_0^1 x \left[\int_0^x (x+y)\,dy\right] dx
        = 2 \int_0^1 x \cdot \frac{3}{2}x^2 dx
        = 3 \int_0^1 x^3 dx
        = 3 \cdot \frac{1}{4}
        = \frac{3}{4}.
    \]

    \[
        \E Y
        = \int_0^1 \int_0^x y \cdot 2(x+y)\,dy\,dx.
    \]
    Внутренний интеграл:
    \[
        \int_0^x y \cdot 2(x+y)\,dy
        = \int_0^x (2xy + 2y^2)\,dy
        = \left[x y^2 + \frac{2}{3}y^3\right]_0^x
        = x^3 + \frac{2}{3}x^3
        = \frac{5}{3}x^3.
    \]
    Значит
    \[
        \E Y
        = \int_0^1 \frac{5}{3}x^3\,dx
        = \frac{5}{3} \cdot \frac{1}{4}
        = \frac{5}{12}.
    \]

    \textbf{Квадраты.}
    \[
        \E X^2
        = \int_0^1 \int_0^x x^2 \cdot 2(x+y)\,dy\,dx
        = 2 \int_0^1 x^2 \left[\int_0^x (x+y)\,dy\right] dx
        = 2 \int_0^1 x^2 \cdot \frac{3}{2}x^2 dx
        = 3 \int_0^1 x^4 dx
        = \frac{3}{5}.
    \]

    \[
        \E Y^2
        = \int_0^1 \int_0^x y^2 \cdot 2(x+y)\,dy\,dx
        = \int_0^1 \left[ \int_0^x (2x y^2 + 2y^3)\,dy \right] dx.
    \]
    Внутренний интеграл:
    \[
        \int_0^x (2x y^2 + 2y^3)\,dy
        = \left[ \frac{2x}{3}y^3 + \frac{1}{2}y^4 \right]_0^x
        = \frac{2x}{3}x^3 + \frac{1}{2}x^4
        = \frac{2}{3}x^4 + \frac{1}{2}x^4
        = \frac{7}{6}x^4.
    \]
    И значит
    \[
        \E Y^2
        = \int_0^1 \frac{7}{6}x^4\,dx
        = \frac{7}{6}\cdot \frac{1}{5}
        = \frac{7}{30}.
    \]

    \textbf{Поперечный момент.}
    \[
        \E[XY]
        = \int_0^1 \int_0^x x y \cdot 2(x+y)\,dy\,dx
        = \int_0^1 \left[ \int_0^x (2x^2 y + 2x y^2)\,dy \right] dx.
    \]
    Внутренний интеграл:
    \[
        \int_0^x (2x^2 y + 2x y^2)\,dy
        = \left[x^2 y^2 + \frac{2x}{3}y^3\right]_0^x
        = x^2 x^2 + \frac{2x}{3} x^3
        = x^4 + \frac{2}{3}x^4
        = \frac{5}{3}x^4.
    \]
    Тогда
    \[
        \E[XY]
        = \int_0^1 \frac{5}{3}x^4\,dx
        = \frac{5}{3}\cdot \frac{1}{5}
        = \frac{1}{3}.
    \]

    Теперь дисперсии и ковариация:
    \[
        \Var(X) = \E X^2 - (\E X)^2
        = \frac{3}{5} - \left(\frac{3}{4}\right)^2
        = \frac{3}{5} - \frac{9}{16}
        = \frac{48}{80} - \frac{45}{80}
        = \frac{3}{80}.
    \]
    \[
        \Var(Y) = \E Y^2 - (\E Y)^2
        = \frac{7}{30} - \left(\frac{5}{12}\right)^2
        = \frac{7}{30} - \frac{25}{144}
        = \frac{168}{720} - \frac{125}{720}
        = \frac{43}{720}.
    \]
    \[
        \Cov(X,Y)
        = \E[XY] - \E X \cdot \E Y
        = \frac{1}{3} - \frac{3}{4}\cdot \frac{5}{12}
        = \frac{1}{3} - \frac{15}{48}
        = \frac{16}{48} - \frac{15}{48}
        = \frac{1}{48}.
    \]

    Коэффициент корреляции:
    \[
        \rho_{X,Y}
        = \frac{\Cov(X,Y)}{\sqrt{\Var(X)\Var(Y)}}
        = \frac{\frac{1}{48}}{\sqrt{\frac{3}{80}\cdot \frac{43}{720}}}
        = \frac{\frac{1}{48}}{\sqrt{\frac{129}{57600}}}
        = \frac{\frac{1}{48}}{\frac{\sqrt{129}}{240}}
        = \frac{240}{48\sqrt{129}}
        = \frac{5}{\sqrt{129}}.
    \]

    \subsection*{(c) Преобразование $U = \dfrac{2Y}{X}$, $V = 2X - Y$}

    \textbf{Обратное преобразование.}
    Решим систему
    \[
        u = \frac{2y}{x}, \qquad v = 2x - y
        \quad \Longrightarrow \quad
        x = \frac{2v}{4-u},\quad
        y = \frac{u v}{4-u}.
    \]

    \textbf{Область значений.}
    Изначально $(x,y)$ лежат в треугольнике $0 \le y < x \le 1$.

    Подставим $x,y$ через $u,v$.

    Условия:
    \[
        x = \frac{2v}{4-u} \ge 0,\quad
        y = \frac{uv}{4-u} \ge 0,\quad
        y < x,\quad
        x \le 1.
    \]

    Из анализа неравенств следует:
    \[
        0 \le u < 2,\qquad
        0 \le v \le \frac{4-u}{2}.
    \]

    (Если $u \ge 2$, то неравенство $y<x$ нарушается; знаки $x,y \ge 0$ также вынуждают $u\ge0$, $v\ge0$, и $x\le1$ даёт верхнюю грань $v\le(4-u)/2$.)

    \textbf{Якобиан.}
    Рассмотрим отображение $(u,v)\mapsto (x,y)$:
    \[
        x(u,v)=\frac{2v}{4-u},\qquad
        y(u,v)=\frac{uv}{4-u}.
    \]
    Якобиан:
    \[
        J =
        \begin{vmatrix}
            \frac{\partial x}{\partial u} & \frac{\partial x}{\partial v}\\[6pt]
            \frac{\partial y}{\partial u} & \frac{\partial y}{\partial v}
        \end{vmatrix}.
    \]
    Вычисления дают
    \[
        \left|\det \frac{\partial(x,y)}{\partial(u,v)}\right|
        = \frac{2v}{(4-u)^2}.
    \]

    \textbf{Совместная плотность $(U,V)$.}
    Так как
    \[
        f_{X,Y}(x,y) = 2(x+y),
    \]
    то при подстановке $x=\frac{2v}{4-u}$, $y=\frac{uv}{4-u}$ имеем
    \[
        x+y = \frac{2v}{4-u} + \frac{uv}{4-u}
        = \frac{v(2+u)}{4-u}.
    \]
    Следовательно,
    \[
        f_{U,V}(u,v)
        = f_{X,Y}(x(u,v),y(u,v))
        \cdot
        \left|\det \frac{\partial(x,y)}{\partial(u,v)}\right|
        = 2 \cdot \frac{v(2+u)}{4-u}
        \cdot
        \frac{2v}{(4-u)^2}
        = \frac{4(2+u)\,v^2}{(4-u)^3}.
    \]

    Итак
    \[
        f_{U,V}(u,v)
        =
        \begin{cases}
            \dfrac{4(2+u)\,v^2}{(4-u)^3}, & 0 \le u < 2,\ 0 \le v \le \dfrac{4-u}{2},\\[10pt]
            0, & \text{иначе.}
        \end{cases}
    \]

    \textbf{Маргинальные плотности.}

    1) Для $U$:
    \[
        f_U(u)
        = \int_0^{(4-u)/2} f_{U,V}(u,v)\,dv
        = \int_0^{(4-u)/2} \frac{4(2+u)\,v^2}{(4-u)^3}\,dv.
    \]
    Интегрируя по $v$,
    \[
        f_U(u)
        = \frac{2+u}{6}, \qquad 0 \le u < 2.
    \]

    2) Для $V$:
    при фиксированном $v$ область по $u$ задаётся неравенствами
    \[
        0 \le u < 2,\qquad
        v \le \frac{4-u}{2}
        \quad\Longleftrightarrow\quad
        u \le 4-2v.
    \]
    Отсюда:

    -- Если $0 \le v \le 1$, то $4-2v \ge 2$, значит $u\in[0,2)$.

    -- Если $1 \le v \le 2$, то $4-2v \le 2$, и $u \in[0,\,4-2v]$.

    При $v>2$ область пуста.

    Тогда
    \[
        f_V(v)
        =
        \begin{cases}
            \displaystyle
            \int_0^{2} \frac{4(2+u)v^2}{(4-u)^3}\,du
            = \frac{5}{4}v^2, & 0 \le v \le 1,\\[12pt]
            \displaystyle
            \int_0^{4-2v} \frac{4(2+u)v^2}{(4-u)^3}\,du
            = \frac{v^2}{4} + 3 - 2v, & 1 \le v \le 2,\\[12pt]
            0, & \text{иначе.}
        \end{cases}
    \]

    \textbf{Являются ли $U$ и $V$ независимыми?}

    Если бы $U$ и $V$ были независимыми, то $f_{U,V}(u,v)=f_U(u)\,f_V(v)$.

    Проверим в точке $(u,v)=(1,1)$:
    \[
        f_{U,V}(1,1)
        = \frac{4(2+1)\cdot 1^2}{(4-1)^3}
        = \frac{12}{27}
        = \frac{4}{9}.
    \]
    \[
        f_U(1) = \frac{2+1}{6} = \frac12,
        \qquad
        f_V(1) = \frac{5}{4}.
    \]
    Тогда
    \[
        f_U(1)\,f_V(1)
        = \frac12 \cdot \frac{5}{4}
        = \frac{5}{8}
        \ne \frac{4}{9}.
    \]
    Следовательно, $U$ и $V$ не независимы.


    \section*{Задача 3}

    Пусть
    \[
        X =
        \begin{pmatrix}
            X_1\\ X_2\\ X_3\\ X_4
        \end{pmatrix}
        \sim
        \mathcal{N}_4(\mu,\Sigma),
        \quad
        \mu =
        \begin{pmatrix}
            1\\2\\-3\\0
        \end{pmatrix},
        \quad
        \Sigma =
        \begin{pmatrix}
            2 & 2 & -1 & -3\\
            2 & 3 & -2 & 1\\
            -1 & -2 & 4 & 0\\
            -3 & 1 & 0 & 5
        \end{pmatrix}.
    \]

    \subsection*{(a) Распределение $X_1 - 2X_3$ и коэффициент корреляции между $X_1$ и $X_3$}

    Обозначим
    \[
        Y = X_1 - 2X_3 = a^\top X,\quad
        a = (1,0,-2,0)^\top.
    \]
    Линейная комбинация многомерной нормали снова нормальна:
    \[
        Y \sim \mathcal{N}(m_Y,\sigma_Y^2),
    \]
    где
    \[
        m_Y = a^\top \mu
        = 1 - 2(-3)
        = 7,
    \]
    \[
        \sigma_Y^2 = a^\top \Sigma a.
    \]
    Вычислим $a^\top \Sigma a$:
    \[
        \sigma_Y^2 = 22.
    \]
    Итого
    \[
        X_1 - 2X_3 \sim \mathcal{N}(7,\,22).
    \]

    Теперь коэффициент корреляции между $X_1$ и $X_3$:
    \[
        \Corr(X_1,X_3)
        = \frac{\Cov(X_1,X_3)}{\sqrt{\Var(X_1)\Var(X_3)}}.
    \]
    Из матрицы $\Sigma$:
    \[
        \Var(X_1)=2,\quad
        \Var(X_3)=4,\quad
        \Cov(X_1,X_3)=-1.
    \]
    Значит
    \[
        \Corr(X_1,X_3)
        = \frac{-1}{\sqrt{2\cdot 4}}
        = -\frac{1}{2\sqrt{2}}.
    \]

    \subsection*{(b) Распределение вектора
        $\bigl(X_2 - 2X_4 + 3X_3 + 1,\; X_1 - 2X_2 + X_4\bigr)^\top$}

    Обозначим
    \[
        Y =
        \begin{pmatrix}
            Y_1\\ Y_2
        \end{pmatrix}
        =
        \begin{pmatrix}
            X_2 - 2X_4 + 3X_3 + 1\\[4pt]
            X_1 - 2X_2 + X_4
        \end{pmatrix}.
    \]
    Это можно записать как
    \[
        Y = A X + b,
        \quad
        A =
        \begin{pmatrix}
            0 & 1 & 3 & -2\\
            1 & -2 & 0 & 1
        \end{pmatrix},
        \quad
        b=
        \begin{pmatrix}
            1\\[2pt]
            0
        \end{pmatrix}.
    \]
    Если $X \sim \mathcal{N}_4(\mu,\Sigma)$, то $Y \sim \mathcal{N}_2(A\mu+b,\; A\Sigma A^\top)$.

    Считаем среднее:
    \[
        A\mu + b
        =
        \begin{pmatrix}
            -6\\
            -3
        \end{pmatrix}.
    \]

    Считаем ковариационную матрицу:
    \[
        A\Sigma A^\top
        =
        \begin{pmatrix}
            43 & 6\\
            6 & 1
        \end{pmatrix}.
    \]

    Следовательно,
    \[
        Y \sim \mathcal{N}_2\!\left(
                                  \begin{pmatrix}
                                      -6\\
                                      -3
                                  \end{pmatrix},
                                  \begin{pmatrix}
                                      43 & 6\\
                                      6 & 1
                                  \end{pmatrix}
        \right).
    \]

    \subsection*{(c) Найти распределение отношения $\dfrac{X_2}{X_4}$}

    Обозначим
    \[
        R = \frac{X_2}{X_4}.
    \]
    Пусть рассматривается двумерный вектор $(X_2,X_4)$, который является двумерной нормальной случайной величиной с параметрами
    \[
        \begin{pmatrix}
            X_2\\ X_4
        \end{pmatrix}
        \sim
        \mathcal{N}_2\!\left(
                           \begin{pmatrix}
                               2\\ 0
                           \end{pmatrix},
                           \;
                           \Sigma_{24}
        \right),
        \quad
        \Sigma_{24}
        =
        \begin{pmatrix}
            3 & 1\\
            1 & 5
        \end{pmatrix}.
    \]
    Его плотность имеет вид
    \[
        f_{X_2,X_4}(x_2,x_4)
        = \frac{1}{2\pi\sqrt{|\Sigma_{24}|}}
        \exp\!\left(
                  -\frac12
                  \begin{pmatrix}
                      x_2-2\\ x_4
                  \end{pmatrix}^{\!\top}
                  \Sigma_{24}^{-1}
                  \begin{pmatrix}
                      x_2-2\\ x_4
                  \end{pmatrix}
        \right),
    \]
    где
    \[
        |\Sigma_{24}| = 14,
        \qquad
        \Sigma_{24}^{-1}
        = \frac{1}{14}
        \begin{pmatrix}
            5 & -1\\
            -1 & 3
        \end{pmatrix}.
    \]

    Чтобы получить плотность $f_R(r)$ случайной величины $R = X_2/X_4$, используем преобразование
    \[
        (x_2,x_4) = (rv, v),\quad v\in\mathbb{R}.
    \]
    Якобиан перехода $(r,v)\mapsto (x_2,x_4)$ равен $|v|$. Тогда
    \[
        f_R(r)
        = \int_{-\infty}^{+\infty} |v|\,
        f_{X_2,X_4}(rv, v)\,dv.
    \]

    Это корректное выражение плотности распределения отношения нормальных компонент (в общем случае явная элементарная формула не упрощается).

    \subsection*{(d) Найти распределение вектора $\bigl(\frac{1}{X_2},\,X_3\bigr)$}

    Обозначим
    \[
        U = \frac{1}{X_2},\qquad V = X_3.
    \]
    Рассмотрим отображение $(u,v)\mapsto(x_2,x_3) = \bigl(\frac{1}{u},\,v\bigr)$, то есть
    \[
        x_2 = \frac{1}{u},\quad x_3 = v.
    \]
    Якобиан:
    \[
        \left|\det \frac{\partial(x_2,x_3)}{\partial(u,v)} \right|
        = \left|\det
              \begin{pmatrix}
                  -\frac{1}{u^2} & 0\\
                  0 & 1
              \end{pmatrix}
        \right|
        = \frac{1}{u^2}.
    \]

    Пара $(X_2,X_3)$ является двумерно нормальной:
    \[
        \begin{pmatrix}
            X_2\\ X_3
        \end{pmatrix}
        \sim
        \mathcal{N}_2\!\left(
                           \begin{pmatrix}
                               2\\ -3
                           \end{pmatrix},
                           \;
                           \Sigma_{23}
        \right),
        \quad
        \Sigma_{23} =
        \begin{pmatrix}
            3 & -2\\
            -2 & 4
        \end{pmatrix}.
    \]
    Её плотность обозначим $f_{X_2,X_3}(x_2,x_3)$.

    Тогда совместная плотность $(U,V)$ равна
    \[
        f_{U,V}(u,v)
        = \frac{1}{u^2}\,
        f_{X_2,X_3}\!\left(\frac{1}{u},\,v\right),
        \quad u\neq 0.
    \]
    (Событие $X_2=0$ имеет нулевую вероятность при непрерывном нормальном распределении, поэтому обращение $u=1/x_2$ корректно п.н.)

    Таким образом задано распределение пары $\left(\frac{1}{X_2}, X_3\right)$ через явную формулу для плотности.

    \subsection*{(e)$^\ast$ Найти распределение $\sqrt{X_2^2 + X_3^2}$}

    Пусть
    \[
        R = \sqrt{X_2^2 + X_3^2}.
    \]
    Это длина вектора $(X_2,X_3)$.

    Плотность $(X_2,X_3)$ нам известна: это двумерная нормальная плотность
    \[
        f_{X_2,X_3}(x_2,x_3)
        = \frac{1}{2\pi\sqrt{|\Sigma_{23}|}}
        \exp\!\left(
                  -\frac12
                  \begin{pmatrix}
                      x_2-2\\ x_3+3
                  \end{pmatrix}^{\!\top}
                  \Sigma_{23}^{-1}
                  \begin{pmatrix}
                      x_2-2\\ x_3+3
                  \end{pmatrix}
        \right).
    \]

    Чтобы получить плотность $f_R(r)$, можно перейти к полярным координатам
    \[
        x_2 = r \cos\theta,\quad
        x_3 = r \sin\theta,
        \quad r \ge 0,\ \theta \in [0,2\pi).
    \]
    Якобиан такого перехода равен $r$. Тогда маргинальная плотность $R$ равна
    \[
        f_R(r)
        = \int_{0}^{2\pi} r\,
        f_{X_2,X_3}\bigl(r\cos\theta,\ r\sin\theta\bigr)\,d\theta,
        \qquad r \ge 0.
    \]

    В общем случае (когда ковариационная матрица не равна $\sigma^2 I$ и вектор не имеет нулевого среднего) это выражение не упрощается в элементарные функции. Тем не менее формула выше корректно задаёт распределение $\sqrt{X_2^2 + X_3^2}$.


\end{document}


\documentclass[12pt]{article}
\usepackage[T2A]{fontenc}
\usepackage[utf8]{inputenc}
\usepackage[russian]{babel}
\usepackage{amsmath,amssymb,mathtools}
\usepackage{geometry}
\geometry{margin=1in}
\usepackage{hyperref}
\newcommand{\PP}{\mathbb{P}}
\newcommand{\C}[2]{\binom{#1}{#2}}
\begin{document}

\begin{center}
{\LARGE Оформленные решения по теории вероятностей}\\[4pt]
{\small (карточная задача; байесовская задача; условная вероятность)}
\end{center}

\section*{Задача 1. Пять карт из колоды 52: ровно 3 красные, ровно 1~червa, не больше 2 тузов}

\textbf{Шаг 1. Без ограничения по тузам.}
Требование ``ровно 3 красные'' и ``ровно 1 черва'' означает композицию мастей:
$1$ черва, $2$ бубны, $2$ чёрные. Число пятёрок карт без учёта тузов:
\[N_0=\C{13}{1}\C{13}{2}\C{26}{2}=329\,550.\]

\textbf{Шаг 2. Вычтем ``плохие'' случаи с $\ge 3$ тузами.}

\emph{(i) Ровно 4 туза.} Тогда пятая карта должна быть бубновой, не тузом: $12$ вариантов.
\[
N_{4\text{A}}=12.
\]

\emph{(ii) Ровно 3 туза.} Разбиваем по отсутствующему тузу.
\begin{itemize}
\item Нет $A\diamondsuit$: есть $A\heartsuit,A\clubsuit,A\spadesuit$; нужно добавить $2$ бубны не тузы: $\C{12}{2}=66$.
\item Нет $A\heartsuit$: есть $A\diamondsuit,A\clubsuit,A\spadesuit$; нужно добавить по одной нетузовой черве и бубне: $12\cdot 12=144$.
\item Нет $A\clubsuit$: есть $A\heartsuit,A\diamondsuit,A\spadesuit$; нужно добавить одну нетузовую бубну и одну нетузовую чёрную (клуб/пика): $12\cdot 24=288$.
\item Нет $A\spadesuit$: симметрично предыдущему: $12\cdot 24=288$.
\end{itemize}
Итого для $3$ тузов: $66+144+288+288=786$.

\smallskip
Суммарно ``плохих'' с $3$ или $4$ тузами: 
\[
N_{\ge3\text{A}}=786+12=798.
\]

\textbf{Шаг 3. Ответ.} Благоприятных пятёрок 
\[
N = N_0-N_{\ge3\text{A}}=329\,550-798=328\,752.
\]
Вероятность
\[
\PP=\frac{N}{\C{52}{5}}=\frac{328\,752}{2\,598\,960}
=\frac{6\,849}{54\,145}\approx 0.12649\;\;(\approx 12.649\%).
\]

\section*{Задача 2. Байесовский расчёт для положительного теста}
Пусть болезнь встречается с вероятностью $p=0.01$. Считаем, что вероятность ошибки у теста одинакова
для обоих типов ошибок: $e=0.025$, то есть чувствительность и специфичность равны $1-e=0.975$.

\subsection*{a)} Вероятность быть действительно больным при положительном результате:
\begin{align*}
\PP(\text{болен}\mid +)
&=\frac{(1-e)\,p}{(1-e)\,p+e\,(1-p)}
=\frac{0.975\cdot 0.01}{0.975\cdot 0.01+0.025\cdot 0.99}\\
&=\frac{0.00975}{0.00975+0.02475}
=\frac{13}{46}\approx 0.2826.
\end{align*}

\subsection*{б)} Требуемая ошибка $e$ для достижения 
$\PP(\text{болен}\mid +)\ge 0.75$:
\begin{align*}
\frac{(1-e)p}{(1-e)p+e(1-p)}&\ge 0.75
\;\Longleftrightarrow\;
(1-e)p\ge 0.75\big[(1-e)p+e(1-p)\big]\\
&\Longleftrightarrow\;
e\le \frac{p}{p+3(1-p)}.
\end{align*}
При $p=0.01$:
\(
e\le \dfrac{0.01}{0.01+3\cdot 0.99}\approx 0.003356.
\)
То есть ошибка теста должна быть не более $\approx 0.336\%$.

\medskip
\textit{Замечание.} Если чувствительность и специфичность различаются, 
в формуле нужно подставить отдельно $\text{Se}$ и $1-\text{Sp}$.

\section*{Задача 3. Условная вероятность с независимыми событиями}
Даны независимые $A,B$, $\PP(A)=0.6$, $\PP(A\cap B)=0.3$.
Найти $\PP\!\big(\overline{A\setminus B}\,\big|\,A\cup B\big)$.

Заметим:
\(
A\setminus B=A\cap \overline B,\ 
\overline{A\setminus B}=\overline{A\cap \overline B}= \overline A\cup B.
\)
Тогда
\begin{align*}
\PP\big(\overline{A\setminus B}\mid A\cup B\big)
&=\frac{\PP\big((\overline A\cup B)\cap (A\cup B)\big)}{\PP(A\cup B)}
=\frac{\PP(B)}{\PP(A\cup B)}.
\end{align*}
По независимости $\PP(B)=\dfrac{\PP(A\cap B)}{\PP(A)}=\dfrac{0.3}{0.6}=0.5$, а
\(
\PP(A\cup B)=\PP(A)+\PP(B)-\PP(A\cap B)=0.6+0.5-0.3=0.8.
\)
Следовательно,
\(
\PP=0.5/0.8=\dfrac{5}{8}=0.625.
\)

\end{document}
